
% -----------------------------------------------
% Domain & Problem
% -----------------------------------------------
Organizations are facing increasing pressure to report Greenhouse Gas (GHG) emissions and optimize their processes for sustainability.
Object-Centric Process Mining (OCPM) provides insight into business processes from multiple perspectives based on automatically logged event data.
However, the application of OCPM for emission analysis is limited by the general unavailability of emission data within Object-Centric Event Logs (OCELs).

% -----------------------------------------------
% Method, Evaluation & Results
% -----------------------------------------------
In this work, we present a method for
% emission analysis based on data in the OCEL~2.0 format.
% This includes
1)\ estimating carbon emissions based on data in the OCEL~2.0 format,
2)\ determining total event emissions to provide event-driven sustainability insights,
and 3)\ allocating event emissions to a set of objects using an allocation rule.

We present a selection of allocation rules
and evaluate their performance on simulated event data.
A graph-based allocation rule is found to produce the best results.
Moreover, dropping resource objects with long lifecycles from the graph further refines the results and significantly reduces runtime.
The OCEAn web application (short for Object-Centric Emission Analysis) is developed based on these findings
and allows for the practical application of our methods to enable further research on process mining for sustainability by enriching OCELs with emission data.
